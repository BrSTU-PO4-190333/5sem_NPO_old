\section{Материалы предварительного проектирования системы}
\subsection{Функциональная схема обработки данных}

\begin{figure}[!htbp]
    \centering
    \includegraphics[height=19cm, width=16cm]
        {_assets/gpia_part2.png}
    \caption{Функциональная схема обработки данных}
\end{figure}

\subsection{Описание картотек}

Картотеки:

\begin{itemize}
    \item Первичные документы \gpiFIO\/a\_pd;
    \item[]\hspace{0pt}
    \item Регистрационный журнал (РЖ) \gpiFIO\/a\_rj;
    \item Книга счетов (КС) \gpiFIO\/a\_ks;
    \item[]\hspace{0pt}
    \item Определение первичных документов \gpiFIO\/a\_opd;
    \item Типовые хозяйственные операции (ТХО) \gpiFIO\/a\_txo;
    \item План счетов (ПС) \gpiFIO\/a\_ps;
    \item Коды аналитического учёта (КАУ) \gpiFIO\/a\_kau;
    \item Виды аналитики \gpiFIO\/a\_va;
    \item[]\hspace{0pt}
    \item Настройки системы \gpiFIO\/a\_nst;
\end{itemize}

\begin{table}[h!p]
    \centering
    \scriptsize
    \caption{Первичные документы \gpiFIO\/a\_pd}
    \begin{tabular}{|p{7cm}|p{7cm}|c|}

\hline
\multicolumn{1}{|c}{\textbf{Реквизит}}
&\multicolumn{1}{|c}{\textbf{Обозначение}}  
&\multicolumn{1}{|p{1.6cm}|}{\textbf{Тип и значность}} 
\\ \hline

поле связи =0                       &\gpiFIO\/a\_pd\_0      &c1     \\ \hline
код документа < --- opd             &\gpiFIO\/a\_pd\_dokk   &c3     \\ \hline
номер документа                     &\gpiFIO\/a\_pd\_dokn   &n5     \\ \hline
дата документа                      &\gpiFIO\/a\_pd\_dokd   &D      \\ \hline
вид аналитики 1 *opd                &\gpiFIO\/a\_pd\_av1    &c3     \\ \hline
тип аналитики 1 =д, к, x            &\gpiFIO\/a\_pd\_avt1   &c1     \\ \hline
аналитика код 1 < --- kau           &\gpiFIO\/a\_pd\_ak1    &c10    \\ \hline
вид аналитики2                      &\gpiFIO\/a\_pd\_av2    &c3     \\ \hline
тип аналитики2                      &\gpiFIO\/a\_pd\_avt2   &c1     \\ \hline
аналитика код2                      &\gpiFIO\/a\_pd\_ak2    &c10    \\ \hline
вид аналитики3                      &\gpiFIO\/a\_pd\_av3    &c3     \\ \hline
тип аналитики3                      &\gpiFIO\/a\_pd\_avt3   &c1     \\ \hline
аналитика код3                      &\gpiFIO\/a\_pd\_ak3    &c10    \\ \hline
операции                            &\gpiFIO\/a\_pd\_to     &c10    \\ \hline
дебет счет *txo                     &\gpiFIO\/a\_pd\_db     &n2     \\ \hline
дебет счет субсчет наименование *txo&\gpiFIO\/a\_pd\_dbn    &c10    \\ \hline
кредит  *txo                        &\gpiFIO\/a\_pd\_kr     &n2     \\ \hline
кредит название *txo                &\gpiFIO\/a\_pd\_krn    &c10    \\ \hline
сумма                               &\gpiFIO\/a\_pd\_rub    &n8     \\ \hline
SAE                                 &\gpiFIO\/a\_pd\_sae    &c10    \\ \hline

    \end{tabular}
\end{table}

\begin{table}[h!p]
    \centering
    \scriptsize
    \caption{Виды аналитики \gpiFIO\/a\_va}
    \begin{tabular}{|p{7cm}|p{7cm}|c|}

\hline
\multicolumn{1}{|c}{\textbf{Реквизит}}
&\multicolumn{1}{|c}{\textbf{Обозначение}}  
&\multicolumn{1}{|p{1.6cm}|}{\textbf{Тип и значность}} 
\\ \hline

поле связи =0                       &\gpiFIO\/a\_va\_0      &c1     \\ \hline
вид аналитики                       &\gpiFIO\/a\_va\_k      &c3     \\ \hline
название вида аналитики             &\gpiFIO\/a\_va\_n      &c15    \\ \hline

    \end{tabular}
\end{table}

\begin{table}[h!p]
    \centering
    \scriptsize
    \caption{Регистрационный журнал (РЖ) \gpiFIO\/a\_rj}
    \begin{tabular}{|p{7cm}|p{7cm}|c|} 

\hline
\multicolumn{1}{|c}{\textbf{Реквизит}}
&\multicolumn{1}{|c}{\textbf{Обозначение}}  
&\multicolumn{1}{|p{1.6cm}|}{\textbf{Тип и значность}} 
\\ \hline

поле связи =0                       &\gpiFIO\/a\_rj\_0      & c1    \\ \hline
дата операции                       &\gpiFIO\/a\_rj\_data   & D     \\ \hline
код оправдательного документа       &\gpiFIO\/a\_rj\_dokk   & c3    \\ \hline
номер документа                     &\gpiFIO\/a\_rj\_dokn   & n5    \\ \hline
дата документа                      &\gpiFIO\/a\_rj\_dokd   & D     \\ \hline
содержание операции                 &\gpiFIO\/a\_rj\_to     & c10   \\ \hline
дебет, счет                         &\gpiFIO\/a\_rj\_db     & n2    \\ \hline
дебет, название                     &\gpiFIO\/a\_rj\_dbn    & c10   \\ \hline
кредит, счет                        &\gpiFIO\/a\_rj\_kr     & n2    \\ \hline
кредит название                     &\gpiFIO\/a\_rj\_krn    & c10   \\ \hline
SAE                                 &\gpiFIO\/a\_rj\_sae    & c10   \\ \hline
Сумма                               &\gpiFIO\/a\_rj\_rub    & n10   \\ \hline

    \end{tabular}
\end{table}

\begin{table}[h!p]
    \centering
    \scriptsize
    \caption{Книга счетов(КС) \gpiFIO\/a\_ks}
    \begin{tabular}{|p{7cm}|p{7cm}|c|}

\hline
\multicolumn{1}{|c}{\textbf{Реквизит}}
&\multicolumn{1}{|c}{\textbf{Обозначение}}  
&\multicolumn{1}{|p{1.6cm}|}{\textbf{Тип и значность}} 
\\ \hline

поле связи =0                       &\gpiFIO\/a\_ks\_0      &c1     \\ \hline
дата операции                       &\gpiFIO\/a\_ks\_data   &D      \\ \hline
код оправдательного документа       &\gpiFIO\/a\_ks\_dokk   &c3     \\ \hline
номер документа                     &\gpiFIO\/a\_ks\_dokn   &n5     \\ \hline
дата документа                      &\gpiFIO\/a\_ks\_dokd   &D      \\ \hline
операции                            &\gpiFIO\/a\_ks\_to     &c10    \\ \hline
счет                                &\gpiFIO\/a\_ks\_s      &n2     \\ \hline
счёт название                       &\gpiFIO\/a\_ks\_sn     &c10    \\ \hline
кор. счёт                           &\gpiFIO\/a\_ks\_ks     &n2     \\ \hline
кор. счет наименование              &\gpiFIO\/a\_ks\_ksn    &c10    \\ \hline
сумма дб                            &\gpiFIO\/a\_ks\_db     &n10    \\ \hline
сумма кр                            &\gpiFIO\/a\_ks\_kr     &n10    \\ \hline
SAE                                 &\gpiFIO\/a\_ks\_sae    &c10    \\ \hline

    \end{tabular}
\end{table}

\begin{table}[h!p]
    \centering
    \scriptsize
    \caption{Определение первичных документов \gpiFIO\/a\_opd}
    \begin{tabular}{|p{7cm}|p{7cm}|c|}

\hline
\multicolumn{1}{|c}{\textbf{Реквизит}}
&\multicolumn{1}{|c}{\textbf{Обозначение}}  
&\multicolumn{1}{|p{1.6cm}|}{\textbf{Тип и значность}} 
\\ \hline

поле связи =0                       &\gpiFIO\/a\_opd\_0     &c1     \\ \hline
код документа                       &\gpiFIO\/a\_opd\_k     &c3     \\ \hline
наименование документа              &\gpiFIO\/a\_opd\_n     &c10    \\ \hline
вид аналитики 1 < --- va            &\gpiFIO\/a\_opd\_av1   &c3     \\ \hline
тип аналитики 1 =д, к, x            &\gpiFIO\/a\_opd\_avt1  &c1     \\ \hline
виды аналитики 2                    &\gpiFIO\/a\_opd\_av2   &c3     \\ \hline
тип аналитики 2                     &\gpiFIO\/a\_opd\_avt2  &c1     \\ \hline
вид аналитики 3                     &\gpiFIO\/a\_opd\_av3   &c3     \\ \hline
тип аналитики 2                     &\gpiFIO\/a\_opd\_avt3  &c1     \\ \hline

    \end{tabular}
\end{table}

\begin{table}[h!p]
    \centering
    \scriptsize
    \caption{Типовые хозяйственные операции(ТХО) \gpiFIO\/a\_txo}
    \begin{tabular}{|p{7cm}|p{7cm}|c|}

\hline
\multicolumn{1}{|c}{\textbf{Реквизит}}
&\multicolumn{1}{|c}{\textbf{Обозначение}}  
&\multicolumn{1}{|p{1.6cm}|}{\textbf{Тип и значность}} 
\\ \hline

поле связи =0                       &\gpiFIO\/a\_txo\_0     &c1     \\ \hline
код документа < ---  opd            &\gpiFIO\/a\_txo\_dokk  &c3     \\ \hline
Операции                            &\gpiFIO\/a\_txo\_k     &c10    \\ \hline
дебет, счёт < --- ps\_1             &\gpiFIO\/a\_txo\_db    &n2     \\ \hline
дебет, название * ps\_1             &\gpiFIO\/a\_txo\_dbn   &c10    \\ \hline
кредит < --- ps\_2                  &\gpiFIO\/a\_txo\_kr    &n2     \\ \hline
кредит, название * ps\_2            &\gpiFIO\/a\_txo\_krn   &c10    \\ \hline
SAE                                 &\gpiFIO\/a\_txo\_sae   &c10    \\ \hline

    \end{tabular}
\end{table}

\begin{table}[h!p]
    \centering
    \scriptsize
    \caption{План счетов(ПС) \gpiFIO\/a\_ps}
    \begin{tabular}{|p{7cm}|p{7cm}|c|} 

\hline
\multicolumn{1}{|c}{\textbf{Реквизит}}
&\multicolumn{1}{|c}{\textbf{Обозначение}}  
&\multicolumn{1}{|p{1.6cm}|}{\textbf{Тип и значность}} 
\\ \hline

поле связи =0                       &\gpiFIO\/a\_ps\_0      &c1     \\ \hline
счет                                &\gpiFIO\/a\_ps\_s      &n2     \\ \hline
название счета                      &\gpiFIO\/a\_ps\_n      &c10    \\ \hline
тип счета = а, п, x                 &\gpiFIO\/a\_ps\_typ    &c1     \\ \hline
вид аналитики 1 из VA               &\gpiFIO\/a\_ps\_av1    &c3     \\ \hline
вид аналитики 2 из VA               &\gpiFIO\/a\_ps\_av2    &c3     \\ \hline

    \end{tabular}
\end{table}

\begin{table}[h!p]
    \centering
    \scriptsize
    \caption{Коды аналитического учёта(КАУ) \gpiFIO\/a\_kau}
    \begin{tabular}{|p{7cm}|p{7cm}|c|} 

\hline
\multicolumn{1}{|c}{\textbf{Реквизит}}
&\multicolumn{1}{|c}{\textbf{Обозначение}}  
&\multicolumn{1}{|p{1.6cm}|}{\textbf{Тип и значность}} 
\\ \hline

поле связи =0                       &\gpiFIO\/a\_kau\_0     &c1     \\ \hline
вид аналитики                       &\gpiFIO\/a\_kau\_k     &c5     \\ \hline
вид аналитики                       &\gpiFIO\/a\_kau\_n     &c15    \\ \hline

    \end{tabular}
\end{table}

\begin{table}[h!p]
    \centering
    \scriptsize
    \caption{Настройки системы \gpiFIO\/a\_nst}
    \begin{tabular}{|p{7cm}|p{7cm}|c|}

\hline
\multicolumn{1}{|c}{\textbf{Реквизит}}
&\multicolumn{1}{|c}{\textbf{Обозначение}}  
&\multicolumn{1}{|p{1.6cm}|}{\textbf{Тип и значность}} 
\\ \hline

поле связи =0                       &\gpiFIO\/a\_nst\_0         &c1     \\ \hline
дата текущая                        &\gpiFIO\/a\_nst\_datat     &D      \\ \hline
интервал с                          &\gpiFIO\/a\_nst\_datas     &D      \\ \hline
интервал до                         &\gpiFIO\/a\_nst\_datado    &D      \\ \hline
cчёт                                &\gpiFIO\/a\_nst\_s         &n2     \\ \hline
название счёта                      &\gpiFIO\/a\_nst\_sn        &c10    \\ \hline
название фирмы                      &\gpiFIO\/a\_nst\_firma     &c10    \\ \hline

    \end{tabular}
\end{table}

\subsection{Описание работ}

\begin{table}[h!p]
    \centering
    \scriptsize
    \caption{Описание работ}
    \begin{tabular}{|p{6cm}|p{11cm}|} 

% = = = = = = = = = =

\hline
\multicolumn{1}{|c}{\textbf{Группа работ}}
&\multicolumn{1}{|c|}{\textbf{Работы}}
\\ \hline

% = = = = = = = = = =

Формирование и разноска первичных \par
документов \par
\hspace{0pt} \par
\textbf{\gpiFIO\/a\_Документы}
&
- \gpiFIO\/a\_Ввод текущей даты \par
- \gpiFIO\/a\_Ввод и разноска первичных документов
\\ \hline

% = = = = = = = = = =

Работа с регистрационным журналом \par
\hspace{0pt} \par
\textbf{\gpiFIO\/a\_РЖ}
&
- \gpiFIO\/a\_Просмотр РЖ \par
- \gpiFIO\/a\_Просмотр РЖ (запрос) \par
- \gpiFIO\/a\_Формирование книги счетов из рег. журнала \par
- \gpiFIO\/a\_Просмотр КС \par
- \gpiFIO\/a\_Просмотр КС (запрос) \par
- \gpiFIO\/a\_Печать книги счетов
\\ \hline

% = = = = = = = = = =

Формирование балансовой отчетности \par
\hspace{0pt} \par
\textbf{\gpiFIO\/a\_БО}
&
- \gpiFIO\/a\_Опеделение форм \par
- \gpiFIO\/a\_Оборотно-сальдовая ведомость за период... по счету... \par
- \gpiFIO\/a\_Балансовая ведомость по счету за период... по счету... \par
- \gpiFIO\/a\_Журнал-ордер за период... по счету...
\\ \hline

% = = = = = = = = = =

Сопровождение картотек-справочников \par
\hspace{0pt} \par
\textbf{\gpiFIO\/a\_Картотеки}
&
- \gpiFIO\/a\_Определение первичных документов \par
- \gpiFIO\/a\_Типовые хозяйственные операции(ТХО) \par
- \gpiFIO\/a\_План счетов(ПС) \par
- \gpiFIO\/a\_Виды аналитики \par
- \gpiFIO\/a\_Коды аналитического учёта(КАУ) \par
- \gpiFIO\/a\_Настройка АРМа
\\ \hline

% = = = = = = = = = =

Ведение архивов \par
\hspace{0pt} \par
\textbf{\gpiFIO\/a\_Архивы}
&
- \gpiFIO\/a\_Копирование системы!!! \par
- \gpiFIO\/a\_Восстановление системы!!!
\\ \hline

% = = = = = = = = = =

Выход из системы \par
\hspace{0pt} \par
\textbf{\gpiFIO\/a\_Выход}
&
- \gpiFIO\/a\_Выход из системы
\\ \hline

% = = = = = = = = = =

    \end{tabular}
\end{table}

\newpage
