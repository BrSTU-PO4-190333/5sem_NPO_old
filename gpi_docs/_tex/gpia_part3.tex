\section{Эксплуатационное описание системы}

\subsection{Описание картотек с использованием инструментальной среды}

Картотеки:

\begin{itemize}
    \item Первичные документы gpia\_pd;
    \item[] \hspace{0pt}
    \item Регистрационный журнал (РЖ) gpia\_rj;
    \item Книга счетов (КС) gpia\_ks;
    \item[] \hspace{0pt}
    \item Определение первичных документов gpia\_opd;
    \item Типовые хозяйственные операции (ТХО) gpia\_txo;
    \item План счетов (ПС) gpia\_ps;
    \item Коды аналитического учёта (КАУ) gpia\_kau;
    \item Виды аналитики gpia\_va;
    \item[] \hspace{0pt}
    \item Настройки системы gpia\_nst;
\end{itemize}

\begin{figure}[!htb]
    \centering
    \caption{Первичные документы gpia\_pd}
    \label{fig:gpia_pd}
\end{figure}

\begin{figure}[!htb]
    \centering
    \caption{Регистрационный журнал (РЖ) gpia\_rj}
    \label{fig:gpia_rj}
\end{figure}

\begin{figure}[!htb]
    \centering
    \caption{Книга счетов gpia\_ks}
    \label{fig:gpia_ks}
\end{figure}

\begin{figure}[!htb]
    \centering
    \caption{Определение первичных документов gpia\_opd}
    \label{fig:gpia_opd}
\end{figure}

\begin{figure}[!htb]
    \centering
    \caption{Коды аналитического учёта gpia\_kau}
    \label{fig:gpia_kau}
\end{figure}

\begin{figure}[!htb]
    \centering
    \caption{Типовые хозяйственные операции(ТХО) gpia\_txo}
    \label{fig:gpia_txo}
\end{figure}

\begin{figure}[!htb]
    \centering
    \caption{План счетов(ПС) gpia\_ps}
    \label{fig:gpia_ps}
\end{figure}

\begin{figure}[!htb]
    \centering
    \caption{Виды аналитики gpia\_va}
    \label{fig:gpia_va}
\end{figure}

\begin{figure}[!htb]
    \centering
    \caption{Настройка системы gpia\_nst}
    \label{fig:gpia_nst}
\end{figure}

\newpage

\subsection{Скриншоты меню}

Работы

\begin{itemize}
    \item gpia\_ Формирование и разноска первичных документов.
    \item gpia\_ Работа с РЖ.
    \item gpia\_ Формирование балансовой отчётности.
    \item gpia\_ Сопровождение картотек.
    \item gpia\_ Ведение архивов.
    \item gpia\_ Выход из системы.
\end{itemize}

\begin{figure}[!htb]
    \centering
    \caption{gpia\_ Формирование и разноска первичных документов}
\end{figure}

\begin{figure}[!htb]
    \centering
    \caption{gpia\_ Работа с РЖ}
\end{figure}

\begin{figure}[!htb]
    \centering
    \caption{gpia\_ Формирование балансовой отчётности}
\end{figure}

\begin{figure}[!htb]
    \centering
    \caption{gpia\_ Сопровождение картотек}
\end{figure}

\begin{figure}[!htb]
    \centering
    \caption{gpia\_ Ведение архивов}
\end{figure}

\begin{figure}[!htb]
    \centering
    \caption{gpia\_ Выход из системы}
\end{figure}

\newpage

\subsection{Скриншоты картотек}

Картотеки:

\begin{itemize}
\item Формирование и разноска первичного документа gpia\_pd;
\item[] \hspace{0pt}
\item Просмотр Регистрационный журнал (РЖ) gpia\_rj;
\item Просмотр Книга счетов (КС) gpia\_ks;
\item[] \hspace{0pt}
\item Определение первичных документов gpia\_opd;
\item Типовые хозяйственные операции (ТХО) gpia\_txo;
\item План счетов (ПС) gpia\_ps;
\item Коды аналитического учёта (КАУ) gpia\_kau;
\item Виды аналитики gpia\_va;
\item[] \hspace{0pt}
\item Ввод текущей даты gpia\_nst\_st;
\item Определение отчетных форм gpia\_nst\_si;
\item Настройка АРМа gpia\_nst\_sf;
\end{itemize}

\begin{figure}[!htb]
    \centering
    \caption{Формирование и разноска первичных документов gpia\_pd\_sx}
    \label{fig:gpia_pd_sx}
\end{figure}

\begin{figure}[!htb]
    \centering
    \caption{Просмотр Регистрационного журнала gpia\_rj}
    \label{fig:gpia_rj}
\end{figure}

\begin{figure}[!htb]
    \centering
    \caption{Книга счетов (КС) gpia\_ks\_sr}
    \label{fig:gpia_ks_sr}
\end{figure}

\begin{figure}[!htb]
    \centering
    \caption{Определение первичных документов gpia\_opd}
    \label{fig:gpia_opd}
\end{figure}

\begin{figure}[!htb]
    \centering
    \caption{Типовые хозяйственные операции (ТХО) gpia\_txo}
    \label{fig:gpia_txo}
\end{figure}

\begin{figure}[!htb]
    \centering
    \caption{Коды аналитического учёта (КАУ) gpia\_kau}
    \label{fig:gpia_kau}
\end{figure}

\begin{figure}[!htb]
    \centering
    \caption{Виды аналитики gpia\_va}
    \label{fig:gpia_va}
\end{figure}

\begin{figure}[!htb]
    \centering
    \caption{Ввод текущей даты gpia\_nst\_st}
    \label{fig:gpia_nst_st}
\end{figure}

\begin{figure}[!htb]
    \centering
    \caption{Настройка АРМа gpia\_nst\_sf}
    \label{fig:gpia_nst_sf}
\end{figure}

\begin{figure}[!htb]
    \centering
    \caption{План счетов (ПС) gpia\_ps\_sx}
    \label{fig:gpia_ps_sx}
\end{figure}

\begin{figure}[!htb]
    \centering
    \caption{Определение отчётных форм gpia\_nst\_si}
    \label{fig:gpia_nst_si}
\end{figure}

\newpage

\subsection{Образцы печатных форм}

\begin{itemize}
    \item Оборотно-сальдовая ведомость за период... по счету... gpia\_ks\_qo
    \item Балансовая ведомость за период... по счету...  gpia\_ks\_qv
    \item Журнал-ордер за период... по счету... gpia\_ks\_qj
    \item Книга счетов gpia\_ks\_qx
\end{itemize}

\begin{figure}[!htb]
    \centering
    \caption{Оборотно-сальдовая ведомость за период... по счету... gpia\_ks\_qo}
    \label{fig:gpia_ks_qo}
\end{figure}

\begin{figure}[!htb]
    \centering
    \caption{Балансовая ведомость за период... по счету... gpia\_ks\_qv}
    \label{fig:gpia_ks_qv}
\end{figure}

\begin{figure}[!htb]
    \centering
    \caption{Журнал-ордер за период... по счету... gpia\_ks\_qj}
    \label{fig:gpia_ks_qj}
\end{figure}

\begin{figure}[!htb]
    \centering
    \caption{Книга счетов  gpia\_ks\_qx}
    \label{fig:gpia_ks_qx}
\end{figure}