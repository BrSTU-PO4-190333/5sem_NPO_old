\section{Материалы предварительного проектирования системы}
\subsection{Функциональная схема обработки данных}

\begin{figure}[!htb]
    \centering
    \includegraphics[height=19cm, width=18cm]
        {_assets/gpiq_part2.png}
    \caption{Функциональная схема обработки данных}
\end{figure}

\subsection{Описание картотек}

Картотеки:

\begin{itemize}
    \item Первичные документы \gpiFIO\/q\_ps\_s;
    \item[] \hspace{0pt}
    \item Регистрационный журнал (РЖ) \gpiFIO\/q\_rj\_s;
    \item Книга счетов (КС) \gpiFIO\/q\_ks;
    \item[] \hspace{0pt}  
    \item Типовые хозяйственные операции (ТХО) \gpiFIO\/q\_txo\_s;
    \item[] \hspace{0pt}     
    \item Настройки системы \gpiFIO\/q\_nst\_s:
    \item Настройки интервалов счетов;
    \item Настройки текущей даты.
\end{itemize}

\begin{table}[h!p]
    \centering
    \scriptsize
    \caption{Первичные документы \gpiFIO\/q\_pd}
    \begin{tabular}{|p{7cm}|p{7cm}|c|}

\hline
\multicolumn{1}{|c}{\textbf{Реквизит}}
&\multicolumn{1}{|c}{\textbf{Обозначение}}  
&\multicolumn{1}{|p{1.6cm}|}{\textbf{Тип и значность}} 
\\ \hline

код документа                       &\gpiFIO\/q\_pd\_dokk   &varchar(4)     \\ \hline
номер документа                     &\gpiFIO\/q\_pd\_dokn   &int            \\ \hline
дата документа                      &\gpiFIO\/q\_pd\_dokd   &date           \\ \hline
операции                            &\gpiFIO\/q\_pd\_to     &varchar(10)    \\ \hline
дебет счет *txo                     &\gpiFIO\/q\_pd\_db     &int            \\ \hline
дебет счет субсчет наименование *txo&\gpiFIO\/q\_pd\_dbn    &varchar(10)    \\ \hline
кредит *txo                         &\gpiFIO\/q\_pd\_kr     &int            \\ \hline
кредит название *txo                &\gpiFIO\/q\_pd\_krn    &varchar(10)    \\ \hline
сумма                               &\gpiFIO\/q\_pd\_rub    &int            \\ \hline

    \end{tabular}
\end{table}

\begin{table}[h!p]
    \centering
    \scriptsize
    \caption{Регистрационный журнал (РЖ) \gpiFIO\/q\_rj}
    \begin{tabular}{|p{7cm}|p{7cm}|c|}

\hline
\multicolumn{1}{|c}{\textbf{Реквизит}}
&\multicolumn{1}{|c}{\textbf{Обозначение}}  
&\multicolumn{1}{|p{1.6cm}|}{\textbf{Тип и значность}} 
\\ \hline

дата операции                       &\gpiFIO\/q\_rj\_data   &date           \\ \hline
код оправдательного документа       &\gpiFIO\/q\_rj\_dokk   &varchar(7)     \\ \hline
номер документа                     &\gpiFIO\/q\_rj\_dokn   &int            \\ \hline
дата документа                      &\gpiFIO\/q\_rj\_dokd   &date           \\ \hline
содержание операции                 &\gpiFIO\/q\_rj\_to     &varchar(50)    \\ \hline
дебет, счет                         &\gpiFIO\/q\_rj\_db     &int            \\ \hline
дебет, название                     &\gpiFIO\/q\_rj\_dbn    &varchar(10)    \\ \hline
кредит, счет                        &\gpiFIO\/q\_rj\_kr     &int            \\ \hline
кредит название                     &\gpiFIO\/q\_rj\_krn    &varchar(10)    \\ \hline
Сумма                               &\gpiFIO\/q\_rj\_rub    &int            \\ \hline

    \end{tabular}
\end{table}

\begin{table}[h!p]
    \centering
    \scriptsize
    \caption{Книга счетов(КС) \gpiFIO\/q\_ks}
    \begin{tabular}{|p{7cm}|p{7cm}|c|}

\hline
\multicolumn{1}{|c}{\textbf{Реквизит}}
&\multicolumn{1}{|c}{\textbf{Обозначение}}  
&\multicolumn{1}{|p{1.6cm}|}{\textbf{Тип и значность}} 
\\ \hline

дата операции                       &\gpiFIO\/q\_ks\_data   &date           \\ \hline
код оправдательного документа       &\gpiFIO\/q\_ks\_dokk   &varchar(4)     \\ \hline
номер документа                     &\gpiFIO\/q\_ks\_dokn   &int            \\ \hline
дата документа                      &\gpiFIO\/q\_ks\_dokd   &date           \\ \hline
операции                            &\gpiFIO\/q\_ks\_to     &varchar(50)    \\ \hline
счет                                &\gpiFIO\/q\_ks\_s      &int            \\ \hline
счёт название                       &\gpiFIO\/q\_ks\_sn     &varchar(10)    \\ \hline
кор. счёт                           &\gpiFIO\/q\_ks\_ks     &int            \\ \hline
кор. счет наименование              &\gpiFIO\/q\_ks\_ksn    &varchar(10)    \\ \hline
сумма дб                            &\gpiFIO\/q\_ks\_rubdb  &int            \\ \hline
сумма кр                            &\gpiFIO\/q\_ks\_rubkr  &int            \\ \hline

    \end{tabular}
\end{table}

\begin{table}[h!p]
    \centering
    \scriptsize
    \caption{Типовые хозяйственные операции(ТХО) \gpiFIO\/q\_txo}
    \begin{tabular}{|p{7cm}|p{7cm}|c|}

\hline
\multicolumn{1}{|c}{\textbf{Реквизит}}
&\multicolumn{1}{|c}{\textbf{Обозначение}}  
&\multicolumn{1}{|p{1.6cm}|}{\textbf{Тип и значность}} 
\\ \hline

код документа                       &\gpiFIO\/q\_txo\_dokk  &varchar(4)     \\ \hline
операции                            &\gpiFIO\/q\_txo\_k     &varchar(10)    \\ \hline
дебет, счёт                         &\gpiFIO\/q\_txo\_db    &int            \\ \hline
дебет, название                     &\gpiFIO\/q\_txo\_dbn   &varchar(10)    \\ \hline
кредит                              &\gpiFIO\/q\_txo\_kr    &int            \\ \hline
кредит, название                    &\gpiFIO\/q\_txo\_krn   &varchar(10)    \\ \hline

    \end{tabular}
\end{table}

\begin{table}[h!p]
    \centering
    \scriptsize
    \caption{Настройки системы \gpiFIO\/q\_nst}
    \begin{tabular}{|p{7cm}|p{7cm}|c|}

\hline
\multicolumn{1}{|c}{\textbf{Реквизит}}
&\multicolumn{1}{|c}{\textbf{Обозначение}}  
&\multicolumn{1}{|p{1.6cm}|}{\textbf{Тип и значность}} 
\\ \hline

дата текущая                        &\gpiFIO\/q\_nst\_datat &date           \\ \hline
интервал с                          &\gpiFIO\/q\_nst\_datas &date           \\ \hline
интервал до                         &\gpiFIO\/q\_nst\_datado&date           \\ \hline
cчёт                                &\gpiFIO\/q\_nst\_s     &int            \\ \hline
название счёта                      &\gpiFIO\/q\_nst\_sn    &varchar(10)    \\ \hline
название фирмы                      &\gpiFIO\/q\_nst\_firma &varchar(10)    \\ \hline

    \end{tabular}
\end{table}

\subsection{Описание работ}

\begin{table}[h!p]
    \centering
    \scriptsize
    \caption{Описание работ}
    \begin{tabular}{|p{6cm}|p{11cm}|} 

% = = = = = = = = = =

\hline
\multicolumn{1}{|c}{\textbf{Группа работ}}
&\multicolumn{1}{|c|}{\textbf{Работы}}
\\ \hline

% = = = = = = = = = =

Формирование и разноска первичных \par
документов \par
\hspace{0pt} \par
\textbf{\gpiFIO\/q\_Документы}
&
- \gpiFIO\/q\_Ввод текущей даты \par
- \gpiFIO\/q\_Ввод и разноска первичных документов (ПД)
\\ \hline

% = = = = = = = = = =

Работа с регистрационным журналом \par
\hspace{0pt} \par
\textbf{\gpiFIO\/q\_РЖ}
&
- \gpiFIO\/q\_Просмотр РЖ \par
- \gpiFIO\/q\_Формирование книги счетов из рег. журнала \par
- \gpiFIO\/q\_Просмотр КС
\\ \hline

% = = = = = = = = = =

Формирование балансовой отчетности \par
\hspace{0pt} \par
\textbf{\gpiFIO\/q\_БО}
&
- \gpiFIO\/q\_Опеделение отчетных форм \par
- \gpiFIO\/q\_Оборотно-сальдовая ведомость  \par
- \gpiFIO\/q\_Балансовая ведомость (заглушка) \par
- \gpiFIO\/q\_Журнал-ордер (заглушка)
\\ \hline

% = = = = = = = = = =

Сопровождение картотек-справочников \par
\hspace{0pt} \par
\textbf{\gpiFIO\/q\_Картотеки}
&
- \gpiFIO\/q\_Типовые хозяйственные операции (ТХО) \par
- \gpiFIO\/q\_Настройка АРМа
\\ \hline

% = = = = = = = = = =

Ведение архивов \par
\hspace{0pt} \par
\textbf{\gpiFIO\/q\_Архивы}
&
- \gpiFIO\/q\_Копирование системы (заглушка) \par
- \gpiFIO\/q\_Восстановление системы (заглушка) \par
\\ \hline

% = = = = = = = = = =

Выход из системы \par
\hspace{0pt} \par
\textbf{\gpiFIO\/q\_Выход}
&
- \gpiFIO\/q\_Выход из системы
\\ \hline

% = = = = = = = = = =

    \end{tabular}
\end{table}

\newpage
