\section{Материалы предварительного проектирования системы}
\subsection{Функциональная схема обработки данных}

\begin{figure}[!htb]
    \centering
    \includegraphics[height=19cm]
        {_assets/gpif_part2.png}
    \caption{Функциональная схема обработки данных}
\end{figure}

\subsection{Описание картотек}

Картотеки:

\begin{itemize}
    \item Первичные документы \gpiFIO\/f\_p;
    \item[]\hspace{0pt}
    \item Регистрационный журнал (РЖ) \gpiFIO\/f\_j;
    \item Книга счетов (КС) \gpiFIO\/f\_k;
    \item[]\hspace{0pt}
    \item Определение первичных документов \gpiFIO\/f\_d;
    \item Типовые хозяйственные операции (ТХО) \gpiFIO\/f\_o;
    \item План счетов (ПС) \gpiFIO\/f\_s;
    \item Коды аналитического учёта (КАУ) \gpiFIO\/f\_a;
    \item Виды аналитики \gpiFIO\/f\_v;
    \item[]\hspace{0pt}
    \item Настройки системы \gpiFIO\/f\_с. 
\end{itemize}

\begin{table}[h!p]
    \centering
    \scriptsize
    \caption{Первичные документы \gpiFIO\/f\_p}
    \begin{tabular}{|l|l|l|} 

                                                                                          \hline
\textbf{Реквизит}                   &\textbf{Обозначение}   &\textbf{Тип и значность}  \\ \hline
поле связи               =0         &\gpiFIO\/f\_p0               &n1                        \\ \hline
код документа    < --- d            &\gpiFIO\/f\_pdokk            &c3                        \\ \hline
номер документа                     &\gpiFIO\/f\_pdokn            &n5                        \\ \hline
дата документа                      &\gpiFIO\/f\_pdokd            &d8                        \\ \hline
вид аналитики 1    *d               &\gpiFIO\/f\_pav1             &c3                        \\ \hline
тип аналитики 1      =д, к, x       &\gpiFIO\/f\_pavt1            &c1                        \\ \hline
аналитика код 1   < --- a           &\gpiFIO\/f\_pak1             &c10                       \\ \hline
вид аналитики2                      &\gpiFIO\/f\_pav2             &c3                        \\ \hline
тип аналитики2                      &\gpiFIO\/f\_pavt2            &c1                        \\ \hline
аналитика код2                      &\gpiFIO\/f\_pak2             &c10                       \\ \hline
вид аналитики3                      &\gpiFIO\/f\_pav3             &c3                        \\ \hline
тип аналитики3                      &\gpiFIO\/f\_pavt3            &c1                        \\ \hline
аналитика код3                      &\gpiFIO\/f\_pak3             &c10                       \\ \hline
сумма                               &\gpiFIO\/f\_prub             &n10                       \\ \hline
операции                            &\gpiFIO\/f\_pto              &c10                       \\ \hline
дебет счет *o                       &\gpiFIO\/f\_pdb              &n2                        \\ \hline
дебет счет субсчет наименование *o  &\gpiFIO\/f\_pdbn             &c10                       \\ \hline
кредит  *o                          &\gpiFIO\/f\_pkr              &n2                        \\ \hline
кредит название*o                   &\gpiFIO\/f\_pkrn             &c10                       \\ \hline

    \end{tabular}
\end{table}

\begin{table}[h!p]
    \centering
    \scriptsize
    \caption{Регистрационный журнал (РЖ) \gpiFIO\/f\_j}
    \begin{tabular}{|l|l|l|} 

                                                                                       \hline
\textbf{Реквизит}               &\textbf{Обозначение}   &\textbf{Тип и значность}   \\ \hline
поле связи	=0                  &\gpiFIO\/f\_j0               &n1                         \\ \hline
дата операции                   &\gpiFIO\/f\_jdata            &d8                         \\ \hline
код оправдательного документа   &\gpiFIO\/f\_jdokk            &c3                         \\ \hline
номер документа                 &\gpiFIO\/f\_jdokn            &n10                        \\ \hline
дата документа                  &\gpiFIO\/f\_jdokd            &d8                         \\ \hline
содержание операции             &\gpiFIO\/f\_jto              &c10                        \\ \hline
дебет, счет                     &\gpiFIO\/f\_jdb              &n2                         \\ \hline
дебет, название                 &\gpiFIO\/f\_jdbn             &c10                        \\ \hline
кредит, счет                    &\gpiFIO\/f\_jkr              &n2                         \\ \hline
кредит название                 &\gpiFIO\/f\_jkrn             &c10                        \\ \hline
Сумма                           &\gpiFIO\/f\_jrub             &n10                        \\ \hline

    \end{tabular}
\end{table}

\begin{table}[h!p]
    \centering
    \scriptsize
    \caption{Книга счетов(КС) \gpiFIO\/f\_k}
    \begin{tabular}{|l|l|l|} 

                                                                                       \hline
\textbf{Реквизит}               &\textbf{Обозначение}   &\textbf{Тип и значность}   \\ \hline
поле связи  =0                  &\gpiFIO\/f\_k0                &n1                         \\ \hline
дата операции                   &\gpiFIO\/f\_kdata             &d8                         \\ \hline
код оправдательного документа   &\gpiFIO\/f\_kdokk             &c3                         \\ \hline
номер документа                 &\gpiFIO\/f\_kdokn             &n10                        \\ \hline
дата документа                  &\gpiFIO\/f\_kdokd             &d8                         \\ \hline
операции                        &\gpiFIO\/f\_kto               &c10                        \\ \hline
счет                            &\gpiFIO\/f\_ks                &n2                         \\ \hline
счёт название                   &\gpiFIO\/f\_ksn               &c10                        \\ \hline
кор. счёт                       &\gpiFIO\/f\_kks               &n2                         \\ \hline
кор. счет наименование          &\gpiFIO\/f\_kksn              &c10                        \\ \hline
сумма дб                        &\gpiFIO\/f\_kdb               &n10                        \\ \hline
сумма кр                        &\gpiFIO\/f\_kkr               &n10                        \\ \hline

    \end{tabular}
\end{table}

\begin{table}[h!p]
    \centering
    \scriptsize
    \caption{Определение первичных документов \gpiFIO\/f\_d}
    \begin{tabular}{|l|l|l|} 

                                                                                   \hline
\textbf{Реквизит}           &\textbf{Обозначение}   &\textbf{Тип и значность}   \\ \hline
поле связи       =0         &\gpiFIO\/f\_d0               &n1                         \\ \hline
код документа               &\gpiFIO\/f\_dk               &c3                         \\ \hline
наименование документа      &\gpiFIO\/f\_dn               &c10                        \\ \hline
вид аналитики 1  < ---  v   &\gpiFIO\/f\_dav1             &c3                         \\ \hline
тип аналитики 1    =д, к, x &\gpiFIO\/f\_davt1            &c1                         \\ \hline
виды аналитики 2            &\gpiFIO\/f\_dav2             &c3                         \\ \hline
тип аналитики 2             &\gpiFIO\/f\_davt2            &c1                         \\ \hline
вид аналитики 3             &\gpiFIO\/f\_dav3             &c3                         \\ \hline
тип аналитики 2             &\gpiFIO\/f\_davt3            &c1                         \\ \hline

    \end{tabular}
\end{table}

\begin{table}[h!p]
    \centering
    \scriptsize
    \caption{Типовые хозяйственные операции(ТХО) \gpiFIO\/f\_o}
    \begin{tabular}{|l|l|l|} 

                                                                                           \hline
\textbf{Реквизит}                   &\textbf{Обозначение}   &\textbf{Тип и значность}   \\ \hline
поле связи    =0                    &\gpiFIO\/f\_o0               &n1                         \\ \hline
код документа        < ---  d       &\gpiFIO\/f\_odok             &c3                         \\ \hline
Операции                            &\gpiFIO\/f\_ok               &c10                        \\ \hline
дебет, счёт              < --- s\_1 &\gpiFIO\/f\_odb              &n2                         \\ \hline
дебет, название        * s\_1       &\gpiFIO\/f\_odbn             &c10                        \\ \hline
кредит                    < --- s\_2&\gpiFIO\/f\_okr              &n2                         \\ \hline
кредит, название     * s\_2         &\gpiFIO\/f\_okrn             &c10                        \\ \hline

    \end{tabular}
\end{table}

\begin{table}[h!p]
    \centering
    \scriptsize
    \caption{План счетов(ПС) \gpiFIO\/f\_s}
    \begin{tabular}{|l|l|l|} 

                                                                                   \hline
\textbf{Реквизит}           &\textbf{Обозначение}   &\textbf{Тип и значность}   \\ \hline
поле связи        =0        &\gpiFIO\/f\_s0               &n1                         \\ \hline
счет                        &\gpiFIO\/f\_sk               &n2                         \\ \hline
название счета              &\gpiFIO\/f\_sn               &c10                        \\ \hline
тип счета          = а, п, x&\gpiFIO\/f\_styp             &c1                         \\ \hline
вид аналитики 1 из V        &\gpiFIO\/f\_sav1             &c3                         \\ \hline
вид аналитики 2 из V        &\gpiFIO\/f\_sav2             &c3                         \\ \hline

    \end{tabular}
\end{table}

\begin{table}[h!p]
    \centering
    \scriptsize
    \caption{Коды аналитического учёта(КАУ) gpia\_a}
    \begin{tabular}{|l|l|l|} 

                                                                               \hline
\textbf{Реквизит}       &\textbf{Обозначение}   &\textbf{Тип и значность}   \\ \hline
поле связи          =0  &\gpiFIO\/f\_a0               &n1                         \\ \hline
вид аналитики           &\gpiFIO\/f\_av               &c3                         \\ \hline
вид аналитики           &\gpiFIO\/f\_ak               &c10                        \\ \hline

    \end{tabular}
\end{table}

\begin{table}[h!p]
    \centering
    \scriptsize
    \caption{Виды аналитики \gpiFIO\/f\_v}
    \begin{tabular}{|l|l|l|} 

                                                                                   \hline
\textbf{Реквизит}           &\textbf{Обозначение}   &\textbf{Тип и значность}   \\ \hline
поле связи  	  =0        &\gpiFIO\/f\_v0               &n1                         \\ \hline
вид аналитики               &\gpiFIO\/f\_vk               &c3                         \\ \hline
название вида аналитики     &\gpiFIO\/f\_vn               &c10                        \\ \hline

    \end{tabular}
\end{table}

\begin{table}[h!p]
    \centering
    \scriptsize
    \caption{Настройки системы gpia\_c}
    \begin{tabular}{|l|l|l|} 

                                                                               \hline
\textbf{Реквизит}       &\textbf{Обозначение}   &\textbf{Тип и значность}   \\ \hline
поле связи          =0  &\gpiFIO\/f\_c0               &n1                         \\ \hline
дата текущая            &\gpiFIO\/f\_cdatt            &d8                         \\ \hline
интервал с              &\gpiFIO\/f\_cdats            &d8                         \\ \hline
интервал до             &\gpiFIO\/f\_cdatd            &d8                         \\ \hline
cчёт                    &\gpiFIO\/f\_cs               &n2                         \\ \hline
название счёта          &\gpiFIO\/f\_csn              &c10                        \\ \hline
название фирмы          &\gpiFIO\/f\_cfirm            &c10                        \\ \hline

    \end{tabular}
\end{table}

\newpage

\subsection{Описание работ}

\begin{table}[h!p]
    \centering
    \scriptsize
    \caption{Описание работ}
    \begin{tabular}{|p{8cm}|p{8cm}|} 

% = = = = = = = = = =

\hline

% = = = = = = = = = =

\textbf{Группа работ}
&
\textbf{Работы}
\\ \hline

% = = = = = = = = = =

Формирование и разноска первичных документов \par
\hspace{0pt} \par
\textbf{\gpiFIO\/f\_документы}
&
- \gpiFIO\/f\_ввод текущей даты \par
- \gpiFIO\/f\_ввод и разноска первичных документов
\\ \hline

% = = = = = = = = = =

Работа с регистрационным журналом \par
\hspace{0pt} \par
\textbf{\gpiFIO\/f\_РЖ}
&
- \gpiFIO\/f\_просмотр РЖ \par
- \gpiFIO\/f\_формирование КС и РЖ \par
- \gpiFIO\/f\_просмотр КС \par
- \gpiFIO\/f\_сформировать КС на печать \par
- \gpiFIO\/f\_просмотр КС для печати
\\ \hline

% = = = = = = = = = =

Формирование балансовой отчетности \par
\hspace{0pt} \par
\textbf{\gpiFIO\/f\_БО}
&
- \gpiFIO\/f\_опеделение отчетных форм \par
- \gpiFIO\/f\_сформировать КС на печать \par
- \gpiFIO\/f\_просмотр КС для печати \par
- \gpiFIO\/f\_сформировать ОСВ \par
- \gpiFIO\/f\_просмотр ОСВ \par
- \gpiFIO\/f\_сформировать Ж-О \par
- \gpiFIO\/f\_просмотр Ж-О \par
- \gpiFIO\/f\_сформировать БВ \par
- \gpiFIO\/f\_просмотр БВ
\\ \hline

% = = = = = = = = = =

Сопровождение картотек-справочников \par
\hspace{0pt} \par
\textbf{\gpiFIO\/f\_картотеки}
&
- \gpiFIO\/f\_определение первичных документов \par
- \gpiFIO\/f\_типовые хозяйственные операции (ТХО) \par  
- \gpiFIO\/f\_план счетов (ПС) \par
- \gpiFIO\/f\_виды аналитики \par 
- \gpiFIO\/f\_коды аналитического учёта (КАУ) \par
- \gpiFIO\/f\_настройка АРМа
\\ \hline

% = = = = = = = = = =

Ведение архивов \par
\hspace{0pt} \par
\textbf{\gpiFIO\/f\_архив}
&
- \gpiFIO\/f\_копия АРМ \par
- \gpiFIO\/f\_восстановление АРМ 
\\ \hline

% = = = = = = = = = =

    \end{tabular}
\end{table}

\newpage
